%%%%%%%%%%%%%%%%%
% This is an sample CV template created using altacv.cls
% (v1.2, 11 February 2020) written by LianTze Lim (liantze@gmail.com). Now compiles with pdfLaTeX, XeLaTeX and LuaLaTeX.
%
%% It may be distributed and/or modified under the
%% conditions of the LaTeX Project Public License, either version 1.3
%% of this license or (at your option) any later version.
%% The latest version of this license is in
%%    http://www.latex-project.org/lppl.txt
%% and version 1.3 or later is part of all distributions of LaTeX
%% version 2003/12/01 or later.
%%%%%%%%%%%%%%%%

%% If you need to pass whatever options to xcolor
\PassOptionsToPackage{dvipsnames}{xcolor}

%% If you are using \orcid or academicons
%% icons, make sure you have the academicons
%% option here, and compile with XeLaTeX
%% or LuaLaTeX.
% \documentclass[10pt,a4paper,academicons]{altacv}

%% Use the "normalphoto" option if you want a normal photo instead of cropped to a circle
% \documentclass[10pt,a4paper,normalphoto]{altacv}

\documentclass[10pt,letter,ragged2e]{altacv}

%% AltaCV uses the fontawesome and academicon fonts
%% and packages.
%% See http://texdoc.net/pkg/fontawesome and http://texdoc.net/pkg/academicons for full list of symbols. You MUST compile with XeLaTeX or LuaLaTeX if you want to use academicons.

% Change the page layout if you need to
\geometry{left=1.25cm,right=1.25cm,top=1.5cm,bottom=1.5cm,columnsep=1.2cm}

% The paracol package lets you typeset columns of text in parallel
\usepackage{paracol}

% Change the font if you want to, depending on whether
% you're using pdflatex or xelatex/lualatex
\ifxetexorluatex
  % If using xelatex or lualatex:
  \setmainfont{Lato}
\else
  % If using pdflatex:
  \usepackage[utf8]{inputenc}
  \usepackage[T1]{fontenc}
  \usepackage[default]{lato}
\fi

% Change the colours if you want to
\definecolor{Mulberry}{HTML}{72243D}
\definecolor{DarkMidnightBlue}{HTML}{00316E}
\definecolor{CetaceanBlue}{HTML}{001540}
\definecolor{SlateGrey}{HTML}{2E2E2E}
\definecolor{LightGrey}{HTML}{666666}
\colorlet{heading}{DarkMidnightBlue}
\colorlet{accent}{CetaceanBlue}
\colorlet{emphasis}{SlateGrey}
\colorlet{body}{LightGrey}

% Change the bullets for itemize and rating marker
% for \cvskill if you want to
\renewcommand{\itemmarker}{{\small\textbullet}}
\renewcommand{\ratingmarker}{\faCircle}

%% sample.bib contains your publications
\addbibresource{sample.bib}

\begin{document}
\name{Nelson Chen}
\tagline{Senior Software Engineer with an interest in rapid prototyping, testing, and productionization}
%% You can add multiple photos on the left or right
% \photoR{2.8cm}{Globe_High}
% \photoL{2.5cm}{Yacht_High,Suitcase_High}
\personalinfo{%
  % Not all of these are required!
  % You can add your own with \printinfo{symbol}{detail}
  \email{nelson@mindflakes.com}
  \phone{626-723-3427}
  \mailaddress{155 East Frye Road Apt 221, Chandler, AZ 85225 / 16168 High Tor Drive, Hacienda Heights, CA}
  \location{Phoenix, AZ / Hacienda Heights, CA}
  \homepage{mindflakes.com}
  \twitter{@crazysim}
  \linkedin{linkedin.com/in/nelsonjchen}
  \github{github.com/nelsonjchen}
  %% You MUST add the academicons option to \documentclass, then compile with LuaLaTeX or XeLaTeX, if you want to use \orcid or other academicons commands.
  % \orcid{orcid.org/0000-0000-0000-0000}
}

\makecvheader

%% Depending on your tastes, you may want to make fonts of itemize environments slightly smaller
\AtBeginEnvironment{itemize}{\small}

%% Set the left/right column width ratio to 6:4.
\columnratio{0.6}

% Start a 2-column paracol. Both the left and right columns will automatically
% break across pages if things get too long.
\begin{paracol}{2}
\cvsection{Experience}


\cvevent{Senior Software Engineer}{Mobivity}{December 2017 - Ongoing}{Chandler, AZ}
\begin{itemize}
\item Refurbished and maintain C++ system call hooking technology for "new" POS systems such as MICROS RES 3700 and more.
\item Prototyped Always-On top C\# Windows WPF Reactive Extensions application for Curbside Pickup and Delivery with simulated backend.
\item Ported PyInstaller-packaged Windows printer interception software and associated tests to CentOS 6 and 7 based systems also using PyInstaller.
\item Configured GitHub Actions for CI/CD for the receipt interception software but for the rest of the company platforms.
\item Prototyped GitLab CI/CD with Kubernetes to test and build ephermeral Windows XP and Windows 7+ POS machines in QEMU-based virtual machines on Google Cloud Platform. 
\item Prototyped reproducible Packer-based Windows 7+ POS simulator virtual machines.
\item Prototyped pytest fixture wrapping QEMU's bundled control library. 
\item Assisted with port of Rails 2 application to Rails 4 by creating Ruby Rspec unit and integration tests, setting up testing Docker images, and configuring GitHub Actions to run test suite.
\item Ported Windows Python 2.6 printer interception software and associated tests to Python 2.7 and 3.7.
\item Configured AppVeyor CI/CD system to compile and package Windows Python software for distribution.
\item Ported legacy developer-system-style installation of deployed point of sale software and runtime to a packaged and single folder and installation footprint with bundled runtime as an additional target using PyInstaller.
\item Continued maintenance of receipt data parser and internals for various Point of Sale systems such as NCR Aloha, Oracle Simphony, SubwayPOS, and many more.
\item Assisted in creation and maintaining of Wix-based Windows MSI installer for receipt interception software.
\item Debugged, produced, and publicly documented previously unknown and novel method to directly load OpenSSL x.509 Base64 "PEM" certificates in the C\# on Windows "HttpClient" without pre-conversion or import of input certificate formats.
\item Produced, and publicly documented Windows PKI code signing for product executables with SHA1 certificates and SHA256 certificates for installation, user space and kernel space loading for Windows 7 RTM/Gold to Windows 10 1809+.
\item Produced Gooey GUI and PyInstaller based tool on multiple platforms to process CSV data into Python unit test harnesses to produce or maintain receipt parsers.
\item Continued support of Django backend and Rails frontend for receipt interception software.
\item Setup AWS Lambda and AWS ElasticSearch-based system for collecting software logs from 30,000 agents running on customer POS systems processing 200GB of logs every day.
\item Modified pre-existing VueJS/Rust Printer simulator for NCR thermal receipt printers for developer use.
\item Ongoing advisement to configuring JIRA to organization workflow.
\item Continued low-level reverse engineering and investigation of various point of sale software and components for integration.
\end{itemize}

\divider

\cvevent{Senior Software Engineer}{Mobivity}{March 2014 - December 2017}{San Diego, CA} 

\begin{itemize}
\item Proposed, completed, and performed transition and import of three company-wide issue tracking systems and two Wikis to Atlassian JIRA and Confluence.
\item Continued maintenance of receipt data parser and internals for various Point of Sales such as NCR Aloha, Oracle Simphony, SubwayPOS, and many more.
\item Assisted in production of a Windows Node-Webkit (predecessor to Electron) and React based installer for receipt interception software.
\item Conceived and implemented low-level Windows function hooking technology called "\textbf{Seamless Capture}" for printer interception software to intercept print data from software without software reconfiguration.
\item Helped create and produce Google App Engine backends for software installer business logic, analysis, and reporting.
\item Continued support of Django backend and Rails frontend for receipt interception software.
\item Supported, extended, and maintained printer stream interception software to 30,000 locations.
\item Developed VueJS/Rust Printer simulator for EPSON and NCR thermal receipt printers for internal developer use.
\item Proposed, completed, and performed adoption of CI/CD for receipt interception software using CircleCI for building the NSIS installers and associated TGZ update for every commit.
\item Proposed, completed, and performed transition and import of existing Atlassian HipChat internal message system to Slack.
\item Continued low-level and high-level reverse engineering and investigation of various point of sale software and components for integration.
\end{itemize}

\divider

\cvevent{Software Engineer}{SmartReceipt}{December 2012 -- March 2014}{Santa Barbara, CA}
\begin{itemize}
\item Contributed to whole infrastructure refactoring with Chef and Salt Stack.
\item Contributed and maintained printer interception receipt data parser , internals of receipt interception software, and tests for software.
\item Developed new rapid Kitchen, Vagrant, and automated testing workflow for Chef infrastructure development and application
\item Produced a Vagrant workflow to organization's Rails 2 application.
\item Produced customer map with static site generators, Google Map APIs and custom styled maps. 
\item Migrated version control practices to Git best practices.
\item Supported, extended, and maintained printer stream interception software in 3,000 locations.
\item Began reverse engineering and investigation of various point of sale software systems and components for integration.
\end{itemize}

\cvevent{Web Architect and Technical Advisor (Volunteer)}{Because of Hope 501(c)(3)}{December 2012 -- Ongoing}{Santa Barbara, CA}
\begin{itemize}
\item Produced VueJS widgets to interactively list students in scholarship program for inclusion on Shopify-based iteration of site
\item Produced complete nanoc static site
\item Setup Heroku-based review system and CI/CD to deploy sites and widgets.
\item Continued maintenance of widgets used on Shopify site,
\item Continued advisement of IT issues regarding emails and newsletter reachability.
\end{itemize}

\cvevent{Technical Committee Vice Chair}{UCSB Associated Students}{2009 -- 2012}{Santa Barbara, CA}
\begin{itemize}

\item Managed technical matters of committee activities including public website and internal software and hardware.
\item Designed, setup, configure, and maintained a VMWare, Linux, Windows, and pfSense-based network, accompanying virtual machines for file sharing and network traffic management, and devised and improved upon workarounds for problematic network issues for LAN Parties.
\item 150+ attendees for every quarter since Fall of 2009 through December 2012.
\end{itemize}


\cvevent{Network Consultant}{UCSB Housing and Residential}{2009 -- 2012}{Santa Barbara, CA}
\begin{itemize}
\item Helped residents in UCSB Residential Housing get their devices online with campus ISP/ResNet. 
\item Developed tools to streamline configuration of these devices for
printing to shared printers and to automatically configure the network. \item Helped physically install switches and hardware in dozens of locations on campus.
\item Maintained internal documentation with regards to networking and procedures maintained. 
\item Assisted users one-on-one in troubleshooting any connectivity
issues both simple and complex. 
\item Produced automated graphical setup tools for all major platforms to assist in configuration of 10,000 user systems with
Printing and Wi-Fi Configuration.
\end{itemize}

\cvsection{Selected Personal Projects and Contributions}

\cvevent{github.com/HackyExtensionsForZoomMeetings/\texttt{BreakoutRoomsBotForZoomMeetings}}{Javascript/Chrome Extensions}{June 2020 -- Ongoing}{Chandler, AZ / Hacienda Heights, CA}

\begin{itemize}
\item Breakout Rooms Bot for Zoom Meetings provides self-switching of Zoom Breakout Rooms for Zoom Meeting Attendees pre-September 2020 Zoom update which added native self-switching. Attendees can self-assign rooms by using chat commands or renaming themselves to a specific format.
\item Reverse engineered Zoom Web Client's React+Redux minified code base and converted Redux store updates to RxJS observables which were then connected together.
\item Reverse engineered undocumented and non-public Websocket protocol to discover fast assignment method.
\item Incorporated fuzzy searching library into bot to make usage much more ergonomic.
\item Implemented as a Chrome extension to allow rapid deployment on the Chrome Web Store.
\item Worked with users and Chrome Developer Console Profiler Debug tooling to get bot performance profile up to standards required for 200+ attendee meetings.
\item After native self-assignment was added, produced GitHub organization and example code for anyone else wishing to add additional "bots" of other nature for their Zoom meetings. Examples of possible use include more Twitch-like amenities for use in classroom settings.
\end{itemize}

\cvevent{github.com/nelsonjchen/\texttt{speedtest-rs}}{Rust}{December 2015 -- Ongoing}{San Diego, CA}

\begin{itemize}
\item Rust-based conversion of popular Python \texttt{speedtest-cli} tool. \texttt{speedtest-cli} and \texttt{speedtest-rs} test internet speed against speedtest.net or ookla.net infrastructure. It's roughly accurate for domestic connections.
\item Utilizes Rust language facilities test to test fixtures and integration test locally including a local loopback server.
\item From 2015 to late 2019, before Ookla released their official native CLI application, \texttt{speedtest-rs} was the only native benchmark application that tested against speedtest.net infrastructure. The Ookla native CLI application is closed source but \texttt{speedtest-rs} is open and FOSS.
\item Unfortunately, this still only tests with the legacy HTTP fallback that its \texttt{speedtest-cli} inspiration uses, but a new socket-based approach and default is planned.
\end{itemize}

\divider

\cvevent{github.com/commaai/\texttt{opendbc}}{Vector DBC}{March 2020 -- Ongoing}{Chandler, AZ}

\begin{itemize}
\item \texttt{opendbc} is an open database of definition files for CAN-bus signals in non-truck vehicles. The database extensively for "openpilot" self driving vehicle project.
\item Contributed reverse engineering of automobile CAN-bus signals for Blind Spot Monitor functionality of the 2020 Toyota Corolla Hatchback and other Toyota Safety Sense 2.0 vehicles for automatic lane change blocking.
\item Automatic lane change blocking from Blind Spot Monitor for Toyota and Hyundai in \texttt{openpilot} landed in OpenPilot 7.7 and was released in July 2020.
\end{itemize}



\divider

\cvevent{github.com/nelsonjchen/\texttt{reverse-rdp-windows-github-actions}}{GitHub Actions/PowerShell/Windows Administration}{December 2019}{San Diego, CA}

\begin{itemize}
\item \texttt{reverse-rdp-windows-github-actions} is a GitHub Actions demo/instruction to reverse RDP into a Windows GitHub Actions build box. This is extremely useful for build troubleshooting on GitHub Actions. A similar first-party functionality on AppVeyor inspired this tool.
\item Starred on GitHub by many senior developers in the Docker community.
\end{itemize}

\divider

\cvevent{github.com/nelsonjchen/\texttt{smashgg\_bracket\_tv}}{TypeScript/Chrome Dev Console}{June 2019}{Hacienda Heights, CA}

\begin{itemize}
\item \texttt{smashgg\_bracket\_tv} is a Typescript project that generates a JavaScript snippet that can be copy and pasted into the Chrome Developer Console on a smash.gg tournament bracket to automatically reformat the Desktop bracket display for a large TV "jumbotron" display. Functions are also added to the console scope which organizers can use remotely by passing into Chrome command line arguments to enable remote connection of a Chrome instance from another Chrome instance on another laptop.
\end{itemize}

\divider

\cvevent{github.com/nelsonjchen/\{\texttt{qli-installer},\texttt{aqtinstall}\}}{Azure-DevOps/Qt Framework}{May 2019 -- December 2019}{Chandler, AZ}

\begin{itemize}
\item \texttt{qli-installer} is a Qt installer that installs Qt dependencies without the official Qt installer. This is a fork of the original qli-installer that adds parallel downloading.
\item Superceded by \texttt{aqtinstaller} of which I contributed a CI system setup to to test building Qt applications on Windows, Mac, and Linux across a wide matrix.
\end{itemize}

\divider

\cvevent{github.com/debauchee/\texttt{barrier}}{Azure-DevOps/Qt Framework}{May 2019 -- July 2019}{Chandler, AZ}

\begin{itemize}
\item \texttt{barrier} is an open source fork of the popular KVM tool "Synergy" before Synergy went closed-source.
\item Contributed Azure DevOps pipeline and build system for Windows build. This was done before the general availability of GitHub Actions.
\item Used as a real test bed for QT CLI installers setups.
\end{itemize}

\divider

\cvevent{github.com/commaai/\texttt{PassShout}}{Kotlin on Android}{Feb 2019 -- Ongoing}{Chandler, AZ}

\begin{itemize}
\item \texttt{PassShout} is a bespoke accessibility service built for Android devices running Eventbrite's Organizer app. A subjective flaw with Eventbrite's Organizer app is that the type of ticket scanned is too small on the screen. When ticket QR codes are scanned, the service will read the type of the ticket scanned out loud to make it more clear what was scanned. Additionally, the app respects Android's audio focus and is able to be used with "music" or a loud looping announcement for attendees being played by controlling those on the device. This is useful for contiuously reminding guests to get out their tickets while the badge handler is moving down the line.
\item Was to be run in 2020's convention but due to COVID-19 the convention was postponed and this app will have its day in 2021 or 2022, maybe.
\end{itemize}

\divider

\cvevent{github.com/nelsonjchen/\texttt{nc-qemu}}{C/Make/Azure Devops}{December 2018}{Chandler, AZ}

\begin{itemize}
\item \texttt{nc-qemu} is a fork of the QEMU hardware virtualizer that builds on Azure DevOps on a Windows machine to legally include Hyper-V-based virtualization support for Windows users and to document executable steps on how to build QEMU on Windows. 
\end{itemize}

\divider

\cvevent{github.com/nelsonjchen/\texttt{gooey-pyinstaller-demo}}{C/Make/Azure Devops}{August 2018}{Chandler, AZ}

\begin{itemize}
\item \texttt{gooey-pyinstaller-demo} is a tutorial I presented at the local Python meetup in Phoenix to show how to take a simple \texttt{Argparse} Python script all the way to a redistributable GUI application built under CI/CD in 5 minutes with Gooey, PyInstaller, and Azure Pipelines. 
\end{itemize}



\divider

\cvevent{github.com/nelsonjchen/\texttt{scoreboard-retro}}{Firebase/VueJS}{Jan 2018 -- Ongoing}{Hacienda Heights, CA}

\begin{itemize}
\item \texttt{scoreboard-retro} is a rough jumbotron scoreboard made for the various tournaments run during the Southern California Retro Videogame Conventions. The rough scoreboard was made overnight and contains an "administrator" interface that can be used by the monitor at the booth. The public display is heavily animated, customized, and updates nearly instantly thanks to VueJS and Firebase.
\end{itemize}

\divider

\cvevent{github.com/nelsonjchen/\texttt{nix-ptsname\_r-shim}}{Rust}{July 2017 -- Ongoing}{San Diego, CA}

\begin{itemize}
\item \texttt{nix-ptsname\_r-shim} is a lightweight shim for use in the \texttt{nix} Rust crate/library to get the psuedo-terminal name on macOS like  the Linux-specific \texttt{ptsname\_r} without the use of the "unsafe" \texttt{ptsname} function. This is useful if you use an API that opens \texttt{/dev/ptmx} to create a psuedoterminal but does not return a file handle to the slave to run \texttt{ttyname} on.
\end{itemize}

\divider

\cvevent{github.com/becauseofhope/\texttt{boh\_components}}{VueJS/Heroku/Typescript}{April 2017 -- Ongoing}{San Diego, CA}

\begin{itemize}
\item \texttt{boh\_components} is used for the 2nd generation website for Because of Hope, a 501(c)(3) organization.
\item The widgets are meant to be included in an existing design.
\item A small Heroku site and deployment is setup so that component changes can be previewed before deploying to production which is used from Shopify.
\end{itemize}

\divider

\cvevent{github.com/nelsonjchen/\texttt{omf-theme-nelsonjchen}}{Fish Shell}{Jan 2016}{San Diego, CA}

\begin{itemize}
\item \texttt{omf-theme-nelsonjchen} is a port/adaptation of @re5et's oh-my-zsh theme for me to the oh-my-fish framework.
\end{itemize}

\divider

\cvevent{github.com/smbc-rss-plus/\texttt{smbc-rss-plus}}{Python}{July 2015 -- Ongoing}{San Diego, CA / Chandler, AZ}

\begin{itemize}
\item \texttt{smbc-rss-plus} takes the popular "Saturday Morning Breakfast Comic" RSS feed, scrapes every item in the feed, visits the link, extracts the "Red Button" image bonus image, and appends it to every item in the RSS feed.
\item This was originally a conversion of a Yahoo Pipes pipeline (RIP). Later iterations required page scraping as the URL to bonus image was no longer constant and page scraping became required.
\item Runs on Heroku under its cron service uploading to S3.
\item This service continues to run and an informational page can be found at smbc-rss-plus.mindflakes.com
\end{itemize}

\divider

\cvevent{github.com/becauseofhope/\texttt{because\_of\_hope}}{Heroku/AWS S3 Web Hosting/nanoc/Ruby/Zurb Foundation}{October 2012 -- July 2017}{San Diego, CA}

\begin{itemize}
\item \texttt{because\_of\_hope} was used for the 1st generation website for Because of Hope, a 501(c)(3) organization.
\item The site is static and deployment consists of plain HTML and images for hosting on S3.
\item \texttt{nanoc} is a Ruby static site generator and is used to generate the student and entrepreneur sponsorship sections which contain many students and entrepreneurs.
\item Uses Zurb Foundation as foundation for CSS styles
\item Heroku was used for review of the site before deploying to production.
\end{itemize}

\divider

\cvevent{github.com/nelsonjchen/\texttt{cowboystyle}}{Python}{May 2012 -- December 2012}{Santa Barbara, CA}

\begin{itemize}
\item \texttt{cowboystyle} is the source code for the /r/UCSantaBarbara subreddit CSS that is viewable on the now-old reddit.
\item Design is extremely unique amongst all college subreddits at the time and possibly still now. Near-total conversion with prominent sidebar element "breaking" the box while keeping well within the security constraints of Reddit CSS's validator.
\item A standard, night, and dark mode was generated into one CSS for compatibility with a popular Reddit extension that offered a night mode.
\item Night mode was also generated for the purposes of switching the CSS to night mode regardless of browser settings when the campus was under nightfall via a bot.
\item Included a watch script that automatically recompiles the CSS and deploys CSS to a test-subreddit for testing.
\end{itemize}

% use ONLY \newpage if you want to force a page break for
% ONLY the current column
\newpage

%% Switch to the right column. This will now automatically move to the second
%% page if the content is too long.
\switchcolumn

\cvsection{My Life Philosophy}

\begin{quote}
``We can do it together!'
\end{quote}

\cvsection{Strengths}

\cvtag{Git} \cvtag{Python} \cvtag{Linux} \cvtag{GitHub}

\cvtag{Windows} \cvtag{Embedded Systems} 

\cvtag{Reverse Engineering} \cvtag{CI/CD} \cvtag{Tooling}


%% Yeah I didn't spend too much time making all the
%% spacing consistent... sorry. Use \smallskip, \medskip,
%% \bigskip, \vpsace etc to make ajustments.
\medskip

\cvsection{Education}

\cvevent{B.Sc.\ Computer Science}{University of California, Santa Barbara}{September 2008 -- December 2012}{}

% \divider

\cvsection{References}

% \cvref{name}{email}{mailing address}
\cvref{Phuc-Hai Huynh}{Mitek Systems}{phuchai.huynh@gmail.com}{619-414-0079}

\divider

\cvref{David Lee}{Microsoft}{davieweb@hotmail.com}{626-503-5242}

\divider

\cvref{Heng Meng}{Boingo}{hengmeng2009@gmail.com}{626-537-5043}

\divider

More references available upon request.

\end{paracol}


\end{document}
